\documentclass{article}
\usepackage[utf8]{inputenc}

\title{ECE351 Lab1}
\author{David Lowe }
\date{September 11, 2019}

\begin{document}

\maketitle

\section{Introduction}
This lab explored different functions and features of Python that will be used throughout the lab this semester. The different sections of this report will discuss the features that were explored.

\section{Description of Python Features}
\subsection{Section 1.3.1 Variables, Matrices, and Arrays}

Variables are simple to define and don't need to be declared as a type like in other programming languages. Printing text and variables is easy using the print command.

Comments can be used to add descriptions of the code you are creating or to add anything to the code that you want don't want the machine to read as it is executing the code. essentially a \# tells the computer to ignore whatever follows on that line.

Since Python is open source, there are many functions that belong to packages that need to be imported. This is simply done by using the import command.

Lists can be created using the list command. Arrays must be created through numpy. Each row in an array must be in brackets with numbers separating each element of the array.

Larger arrays can be used with numpy using the linspace and arrange commands. This allows you to create the array with each element changing by a specified interval up to a specified value.

You can also call specific elements of an array by using its location in the array (row, column).

Matrices of zeros and ones can be built by using numpy.zeros/ones followed by the number of rows and columns of ones or zeros you want to create.

\subsection{1.3.2 Plotting Functions}

Matplotlib.pyplot allows for the easy creation of plots with lots of customization options. You can create separate figures with multiple subplots if you want to compare similar data or you can create single plots which include multiple data sets. You can add axis labels, titles, legends, add grid lines, and  specify plot colors among other things.

\subsection{1.3.3 Complex Numbers}

Numpy can be used to represent complex numbers using 1j as  $\sqrt{-1}$. Sometimes a nan will be returned when the interpreter doesn't know that you want something represented as a complex number, so you have to include + 0j in the root.

\subsection{1.3.4 Additional Helpful Commands}

There are a number of packages that will be used throughout the semester. This section just provides a list of ones we will definitely use, and others that may be useful.

\subsection{1.3.5 Questions}

\subsubsection{For which course are you most excited in your degree? Which course have you enjoyed the
most so far?}

The courses I've enjoyed the most have been my physics courses. Signals and Systems is shaping up to be my favorite this semester. I'm still trying to decide whether I want to pursue power or E\&M or something else altogether.

\subsubsection{Leave any feedback on the clarity of the purpose, deliverables, and tasks for this lab.}

This was a great lab to help get spun up on Python. I've done a small amount of Python work before, but this was a good refresher and it got me excited to learn more. Since I have done some coding before it wasn't too difficult to understand the code in the tasks portions.




\end{document}
